\section{Conclusion and Critical Discussion}
\label{section:conclusion}
Viewed from a critical perspective, the project fulfilled or substituted all of its requirements and passed all its tests.
The \emph{meteoblue} data is incorporated into the system, as well as the elevation model from \emph{ArcGIS} and the live photographs from \emph{Roundshot}.
In \sectionref{section:impl}, a prediction model was described where data of future forecast would be used to estimate the cloud types, but this was not implemented in the final weather rendering system, as this data was provided by the "clouds-1h" package.
\\
The new system outperformed the previous prototype by a factor of 300. Thus, the use of a \gls{computeshader} was definitely worth the effort.
Compared to state-of-the-art cloud systems, the one made during this project can keep up in terms of customization options, but lacks some realistic details and the ability to render cirrus clouds.
\\
Two additional features have been developed that were not originally planned. 
During the implementation phase of the project, these two features have been identified as missing and have been prioritized accordingly.
One of them is a rain \gls{particlesystem} and the other is the shadow mapping.
Both add a substantial part to the credibility of the rendered scenes.