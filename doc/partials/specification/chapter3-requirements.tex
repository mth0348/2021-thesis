\section{Requirements}
\label{section:requirements}
All requirements are grouped by type. This results in two major groups, which are research and development requirements. Each one of the requirements is derived from a goal listed in \sectionref{section:goals}.

\subsection{Research requirements}

\begin{tabularx}{\textwidth}{|l|X|}
    \hline
    \textbf{Number}     & R.\stepcounter{requirements}\arabic{requirements} \\ \hline
    \textbf{Name}       & Understanding of the basic nature of clouds \\ \hline
    \textbf{Description}& In order to be able to recreate a realistically looking cloud shape, one has to examine and understand the way a cloud is formed first. 
                          This includes the color, density, shape as well other characteristics of clouds. \\ \hline
\end{tabularx}
\vspace{0.8cm}

\noindent\begin{tabularx}{\linewidth}{|l|X|}
    \hline
    \textbf{Number}     & R.\stepcounter{requirements}\arabic{requirements} \\ \hline
    \textbf{Name}       & Understanding of what makes good clouds in games \\ \hline
    \textbf{Description}& Just as for real world clouds, clouds in games must also be examined. 
                          This will lead to a clear perception of what is important when rendering clouds in a simulated environment, opposed to the real world example. \\ \hline
\end{tabularx}
\vspace{0.8cm}

\noindent\begin{tabularx}{\linewidth}{|l|X|}
    \hline
    \textbf{Number}     & R.\stepcounter{requirements}\arabic{requirements} \\ \hline
    \textbf{Name}       & Research  common  methods  and  algorithms  involved  in  rendering  procedural  clouds \\ \hline
    \textbf{Description}& It is required to research and keep records of widely-used and popular approaches for rendering clouds.
                          Those methods and algorithms are to be compared and evaluated.
                          \newline
                          They include at least the following topics: 
                          \begin{itemize}
                              \item volumetric rendering
                              \item procedural noise generation algorithms
                              \item the concept of ray marching
                          \end{itemize}
                          \\ \hline
\end{tabularx}

\pagebreak
\subsection{Development requirements}
\setcounter{requirements}{0}
\label{section:requirements:dev}

\noindent\begin{tabularx}{\linewidth}{|l|X|}
    \hline
    \textbf{Number}     & D.\stepcounter{requirements}\arabic{requirements} \\ \hline
    \textbf{Name}       & Implementing a prototype about volumetric rendering \\ \hline
    \textbf{Description}& To fully comprehend volumetric rendering, it is best to implement a prototype, showcasing how this rendering technique works and how it is built. \\ \hline
\end{tabularx}
\vspace{0.8cm}

\noindent\begin{tabularx}{\linewidth}{|l|X|}
    \hline
    \textbf{Number}     & D.\stepcounter{requirements}\arabic{requirements} \\ \hline
    \textbf{Name}       & Implementing a prototype about noise generation \\ \hline
    \textbf{Description}& It is necessary to understand procedural noise generation in order to achieve "natural randomness" as it appears in nature. That is why a prototype must be built from commonly used noise algorithms. \\ \hline
\end{tabularx}
\vspace{0.8cm}

\noindent\begin{tabularx}{\linewidth}{|l|X|}
    \hline
    \textbf{Number}     & D.\stepcounter{requirements}\arabic{requirements} \\ \hline
    \textbf{Name}       & Implementing a prototype about ray marching \\ \hline
    \textbf{Description}& In shader programming, ray marching is a key technique to determine the surface distance of very complex objects, like a cloud. Thus, to see what ray marching is all about, a prototype must be built. \\ \hline
\end{tabularx}
\vspace{0.8cm}

\noindent\begin{tabularx}{\linewidth}{|l|X|}
    \hline
    \textbf{Number}     & D.\stepcounter{requirements}\arabic{requirements} \\ \hline
    \textbf{Name}       & Evaluation of suitable parameters for a cloud shader \\ \hline
    \textbf{Description}& When creating a shader, some parameters should be made public, so the user can influence the outcome of the shader by only tweaking certain variables. 
                          The goal is to find out which parameters are the most influential on the output of the shader. \\ \hline
\end{tabularx}
\vspace{0.8cm}
