\section{Implementation Approach}
Clouds are comprehensible indicators for telling the weather. 
They offer many visible features to make an rough prediction of the weather conditions, or weather changes to come.
As described in \sectionref{section:clouds:types}, some cloud types only form under specific conditions.
Also, whenever certain clouds are present, \gls{precipitation} is shortly followed, as it is with altostratus clouds.
\\
Those factors allow a prediction of the weather, but for this project, the process is reversed.
The given data is not an image of clouds, but meteorological measurement data, and the desired outcome is not a prediction, but an image of clouds.

\begin{figure}[H]
    \centering
    \begin{minipage}{0.47\linewidth}
        \begin{tikzpicture}
            \tikzset{edge/.style = {-{Latex[length=3mm]},shorten >= -4pt}}
            \tikzset{shortedge/.style = {shorten <=-4pt,shorten >= -4pt}}
            \tikzset{shortshortedge/.style = {shorten <=-5.5pt,shorten >= -5.3pt}}
            \tikzset{shortshortedge2/.style = {shorten <=-4.5pt,shorten >= -4.5pt}}

            % rainy clouds
            \node (cloud1) at (0.5, 2.0) {};
            \node (cloud2) at (0.8, 2.2) {};
            \node (cloud5) at (1.6, 2.0) {};
            \node (cloud3) at (1.0, 1.9) {};
            \node (cloud4) at (1.3, 2.2) {};
            \node[cloud,fill=gray!30,cloud puffs=9, cloud, minimum width=0.5cm, minimum height=0.2cm, align=center, draw] (cloud) at (cloud1) {};
            \node[cloud,fill=gray!30,cloud puffs=9, cloud, minimum width=0.5cm, minimum height=0.2cm, align=center, draw] (cloud) at (cloud2) {};
            \node[cloud,fill=gray!30,cloud puffs=9, cloud, minimum width=0.5cm, minimum height=0.2cm, align=center, draw] (cloud) at (cloud5) {};
            \node[cloud,fill=gray!30,cloud puffs=12, cloud, minimum width=1.0cm, minimum height=0.7cm, align=center, draw] (cloud) at (cloud3) {};
            \node[cloud,fill=gray!30,cloud puffs=12, cloud, minimum width=0.8cm, minimum height=0.5cm, align=center, draw] (cloud) at (cloud4) {};
            \draw[edge] (2.5, 2) -- (4,2);
            \node at (5,2) {rain};

            % clear clouds
            \node (cloud1) at (0.5, 0.0) {};
            \node (cloud2) at (0.9, 0.2) {};
            \node (cloud5) at (1.6, 0.0) {};
            \node[cloud,fill=white,cloud puffs=9, cloud, minimum width=0.5cm, minimum height=0.2cm, align=center, draw] (cloud) at (cloud1) {};
            \node[cloud,fill=white,cloud puffs=9, cloud, minimum width=0.5cm, minimum height=0.2cm, align=center, draw] (cloud) at (cloud2) {};
            \node[cloud,fill=white,cloud puffs=9, cloud, minimum width=0.5cm, minimum height=0.2cm, align=center, draw] (cloud) at (cloud5) {};
            \draw[edge] (2.5, 0) -- (4,0);
            \node at (5.7,0) {fair weather};

        \end{tikzpicture}
        \captionof{figure}{Weather information based on visual data.}
        \label{img:tikz:impl:data1}
    \end{minipage}        
    \hfill
    \begin{minipage}{0.47\linewidth}
        \begin{tikzpicture}
            \tikzset{edge/.style = {-{Latex[length=3mm]},shorten >= -4pt}}
            \tikzset{shortedge/.style = {shorten <=-4pt,shorten >= -4pt}}
            \tikzset{shortshortedge/.style = {shorten <=-5.5pt,shorten >= -5.3pt}}
            \tikzset{shortshortedge2/.style = {shorten <=-4.5pt,shorten >= -4.5pt}}

            % rainy clouds
            \node (cloud1) at (4.5, 2.0) {};
            \node (cloud2) at (4.8, 2.2) {};
            \node (cloud5) at (5.6, 2.0) {};
            \node (cloud3) at (5.0, 1.9) {};
            \node (cloud4) at (5.3, 2.2) {};
            \node[cloud,fill=gray!30,cloud puffs=9, cloud, minimum width=0.5cm, minimum height=0.2cm, align=center, draw] (cloud) at (cloud1) {};
            \node[cloud,fill=gray!30,cloud puffs=9, cloud, minimum width=0.5cm, minimum height=0.2cm, align=center, draw] (cloud) at (cloud2) {};
            \node[cloud,fill=gray!30,cloud puffs=9, cloud, minimum width=0.5cm, minimum height=0.2cm, align=center, draw] (cloud) at (cloud5) {};
            \node[cloud,fill=gray!30,cloud puffs=12, cloud, minimum width=1.0cm, minimum height=0.7cm, align=center, draw] (cloud) at (cloud3) {};
            \node[cloud,fill=gray!30,cloud puffs=12, cloud, minimum width=0.8cm, minimum height=0.5cm, align=center, draw] (cloud) at (cloud4) {};
            \draw[edge] (1.5, 2) -- (3,2);
            \node at (0,2) {rain};

            % clear clouds
            \node (cloud1) at (4.5, 0.0) {};
            \node (cloud2) at (4.9, 0.2) {};
            \node (cloud5) at (5.6, 0.0) {};
            \node[cloud,fill=white,cloud puffs=9, cloud, minimum width=0.5cm, minimum height=0.2cm, align=center, draw] (cloud) at (cloud1) {};
            \node[cloud,fill=white,cloud puffs=9, cloud, minimum width=0.5cm, minimum height=0.2cm, align=center, draw] (cloud) at (cloud2) {};
            \node[cloud,fill=white,cloud puffs=9, cloud, minimum width=0.5cm, minimum height=0.2cm, align=center, draw] (cloud) at (cloud5) {};
            \draw[edge] (2.0, 0) -- (3,0);
            \node at (0.7,0) {fair weather};
            
        \end{tikzpicture}
        \captionof{figure}{Visual construction based on weather information.}
        \label{img:tikz:impl:data2}       
    \end{minipage}
\end{figure}

\noindent
For any given day to render, an implementation would require data from that day but also from the near future of that day.
So, in order to render a cloud image for day $x$, a potential algorithm could look like this.
Note that the listing below describes only an idea and is by no means final or compulsory.

\begin{lstlisting}[language=HLSL, caption=Pseudo-code of cloud render algorithm, label=lst:pseudo:algorithm]
// weather data including 7-day forecast
WeatherData data;

function renderClouds(Day x) {
    if (x > TODAY + 7) throw;

    d1 = data.getDataFor(x);
    d2 = data.getDataFor(x + 1);
    d3 = data.getDataFor(x + 2);

    if (d1.rainy) 
        return renderer.cloudsOnRainyDay();
    if (d2.rainy) 
        return renderer.cloudsBeforeRainyDay();
    // many more checks...
}
\end{lstlisting}

\subsection{Look-Ahead Issue}
The approach as described above relies on having data from a couple of days ahead of time. Assumed that number of days is $t$, then the weather data for day $x$ could only be rendered $t$ days after $x$.
That would mean, for such an approach to work, the weather of today can not be rendered before $t$ days later.
\\
In this case however, the weather measurement data retrieved from \emph{meteoblue} also contains a seven-day weather forecast. 
Given that $t$ is less than or equal to seven and an implementation still produces accurate cloud imagery, it would no longer be an issue.

