\section{Project management}

\subsection{Schedule}
The time frame of the semester spans over exactly 16 weeks. Being worth 12 ECTS points, this project assumes a maximum work load of 22.5 hours per week, resulting in a total of 360 hours. 
\vspace{\baselineskip}
\\
Over the course of the term, the project will be split into four primary task groups, namely organization, research, implementation and finalization.
Put into relation with the duration of the project, the estimated schedule looks like this:
\vspace{\baselineskip}

\begin{ganttchart}[
    vgrid={dotted},
    hgrid={draw=black!50, dotted},
    bar/.append style={fill=lightgray},
    x unit=0.65cm,
    milestone node/.append style={fill=orange}
    ]{1}{16}
    \gantttitle{Work weeks}{16} \\
    \gantttitlelist{1,...,16}{1} \\
    \ganttbar{Organization}{1}{4} \\
    \ganttmilestone{Specification finished}{4} \\
    \ganttmilestone{Expert meeting 1}{4} \\
    \ganttgroup{Documentation}{5}{15} \\
    \ganttbar{Research}{5}{11} \\
    \ganttmilestone{Interview with meteoblue}{6} \\
    \ganttbar{Implementation}{7}{15} \\
    \ganttmilestone{Implementation finished}{15} \\
    \ganttmilestone{Expert meeting 2}{15} \\
    \ganttbar{Finalizing}{16}{16}
\end{ganttchart}

%==============================================================

\clearpage
\subsubsection{Task Groups}
For each task group, the following distribution of time and effort is estimated:
\emptyline
\begin{tabular}{|c|c|}
    \hline
    \textbf{Task group}  & \textbf{Predicted effort}\\ \hline
    Organization        & 10\%                      \\ \hline
    Research            & 35\%                      \\ \hline
    Implementation      & 50\%                      \\ \hline
    Finalizing          & 5\%                       \\ \hline
\end{tabular}
\vspace{\baselineskip}

\noindent
The task groups are defined as follows:

\begin{itemize}
    \item \textbf{Organization} \\
    The first task group focuses on creating and finishing the project specification. This also includes the first meeting with the examination expert, Dr. Eric Dubuis.
    
    \item \textbf{Research} \\
    The research spans over the course of almost two months. 
    It also continues being active during the first half of the implementation.
    This is necessary, as the topics will be further investigated when implementing them, which results in more research.
    Also, a technical interview with the firm \emph{meteoblue} will be scheduled and held during the first couple of weeks of the research task.

    \item \textbf{Implementation} \\
    After researching each relevant topic thoroughly, the implementation can begin. In this task, the weather system will be created in Unity.
    Also, during research and implementation, the documentation will be continuously updated.
\end{itemize}

\subsection{Project Organization}
There are two kind of meetings during the project. They will be thoroughly documented in the project's journal.
Should a physical meeting be impossible for some reason, an online meeting via Microsoft Teams will be held instead.

\subsubsection{Weekly meetings}
A meeting will be held on a weekly basis to discuss the progress of the thesis, possibly arisen issues as well as planned work for the upcoming week.
\emptyline
\noindent\begin{tabular}{|l|l|l|}
    \hline
    \textbf{Name}       & \textbf{Role}         & \textbf{Participation}\\ \hline
    Matthias Thomann    & Author                & Mandatory             \\ \hline
    Prof. Urs Künzler   & Tutor and reviewer    & Mandatory             \\ \hline
\end{tabular}

\subsubsection{Expert meetings}
Additionally, both before the research task begins and after the implementation task has ended, a meeting with the external examination expert will be held.
\emptyline
\noindent\begin{tabular}{|l|l|l|}
    \hline
    \textbf{Name}       & \textbf{Role}         & \textbf{Participation}\\ \hline
    Matthias Thomann    & Author                & Mandatory             \\ \hline
    Dr. Eric Dubuis     & Examination expert    & Mandatory             \\ \hline
    Prof. Urs Künzler   & Tutor and reviewer    & Optional              \\ \hline
\end{tabular}
\emptyline

%==============================================================

\subsection{Project Results}
The project results are the following items:
\begin{itemize}
    \item \textbf{Documentation} \\
    The documentation includes this document as well as the thesis' scientific paper.
    \begin{itemize}
        \item Requirement specification
        \item Thesis paper
    \end{itemize}
    \item \textbf{Implementation of the System} \\
    The Unity project, including all implemented shader code, will be managed and stored in the given GitLab repository \cite{gitlab}. This will also serve as a form of submission for grading.
    \item \textbf{Presentation} \\
    A public presentation will be held on the second last Friday of the term, June 9, 2021.
    \item \textbf{Defense of the Thesis} \\
    The bachelor's thesis defense will be held after the term, on a day between June 21, 2021 and July 14, 2021. The exact date is yet to be announced. 
\end{itemize}

\subsubsection{Submission Terms}
The following items must be submitted.
\\\\
\noindent
\begin{tabular}{|l|l|l|}
    \hline
    \textbf{Item}    & \textbf{Description}                                      & \textbf{Due Date}     \\ \hline
    Specification    & This document                                             & March 19, 2021      \\ \hline
    Book entry       & An advertising one-page description of the thesis         & to be announced       \\ \hline
    Poster           & An advertising poster of the thesis (A1 format)           & June 7, 2021        \\ \hline
    Video clip       & An advertising one-minute video clip of the thesis        & June 17, 2021       \\ \hline
    Thesis           & The thesis paper and all of the source code               & June 17, 2021       \\ \hline
    Thesis print     & The printed thesis including a CD with all source code    & June 21, 2021       \\ \hline
\end{tabular}
\newline
\noindent