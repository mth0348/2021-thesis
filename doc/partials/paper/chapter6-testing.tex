\section{Testing}
\label{section:techimpl:testing}
The bachelor project specification document envisioned the following cases to be tested and validated.
The tables each show the test case and the related results, and whether or not the test has been passed. 

\subsection{External Data Testing}

\subsubsection{Weather Data}
\noindent\begin{tabularx}{\textwidth}{|l|X|}
    \hline
    \textbf{Case}            & T.1 \\ \hline
    \textbf{Test case}       & Weather data \\ \hline
    \textbf{Expected result} & The data from \emph{meteoblue} is incorporated into the weather rendering system. The data directly controls all related variables. \\ \hline
    \hline
    \textbf{Actual result}   & \vspace{-\topsep}\begin{itemize}[label={\checkmark},noitemsep,topsep=0pt,leftmargin=*]
                                   \item \emph{meteoblue} data stored periodically
                                   \item Data read at start of system
                                   \item Data controls the simulation
                               \end{itemize} \\ \hline
    \textbf{Fulfilled?}      & Yes \\ \hline
\end{tabularx}

\subsubsection{Terrain Data}
\noindent\begin{tabularx}{\textwidth}{|l|X|}
    \hline
    \textbf{Case}            & T.2 \\ \hline
    \textbf{Test case}       & Terrain data \\ \hline
    \textbf{Expected result} & The data from \emph{swisstopo} is incorporated into the weather rendering system. The elevation model defines the terrain height map. The aerial images are used for texturing. \\ \hline
    \hline
    \textbf{Actual result}   & \vspace{-\topsep}\begin{itemize}[label={$\times$},noitemsep,topsep=0pt,leftmargin=*]
                                   \item \emph{swisstopo} data not used
                               \end{itemize}
                               \begin{itemize}[label={\checkmark},noitemsep,topsep=0pt,leftmargin=*]
                                    \item Substitution found for elevation model data
                                    \item \emph{ArcGIS} plugin is in use and functional
                                \end{itemize} \\ \hline
    \textbf{Fulfilled?}      & No / Substituted \\ \hline
\end{tabularx}

\subsubsection{Photographic Data}
\noindent\begin{tabularx}{\textwidth}{|l|X|}
    \hline
    \textbf{Case}            & T.3 \\ \hline
    \textbf{Test case}       & Photographic data \\ \hline
    \textbf{Expected result} & There is a feature that allows to overlay the \emph{Roundshot} photograph of the same time and date as the rendered image was created for. \\ \hline
    \hline
    \textbf{Actual result}   & \vspace{-\topsep}\begin{itemize}[label={\checkmark},noitemsep,topsep=0pt,leftmargin=*]
                                   \item \emph{Roundshot} images are stored locally
                                   \item The described feature is included in the \gls{ui} 
                               \end{itemize} \\ \hline
    \textbf{Fulfilled?}      & Yes \\ \hline
\end{tabularx}


\subsection{Functional Testing}

\subsubsection{Code Functionality}
\noindent\begin{tabularx}{\textwidth}{|l|X|}
    \hline
    \textbf{Case}            & T.4 \\ \hline
    \textbf{Test case}       & Code functionality \\ \hline
    \textbf{Expected result} & The code for the weather rendering system compiles and runs without error. \\ \hline
    \hline
    \textbf{Actual result}   & \vspace{-\topsep}\begin{itemize}[label={\checkmark},noitemsep,topsep=0pt,leftmargin=*]
                                   \item The code compiles successfully
                                   \item The code runs error-free
                               \end{itemize} \\ \hline
    \textbf{Fulfilled?}      & Yes \\ \hline
\end{tabularx}

\subsubsection{User Interface}
\noindent\begin{tabularx}{\textwidth}{|l|X|}
    \hline
    \textbf{Case}            & T.5 \\ \hline
    \textbf{Test case}       & User interface \\ \hline
    \textbf{Expected result} & The user is able to switch between the two modes, "real mode" and "play mode". The user is also able to control the weather system over the \gls{ui} accordingly. \\ \hline
    \hline
    \textbf{Actual result}   & \vspace{-\topsep}\begin{itemize}[label={\checkmark},noitemsep,topsep=0pt,leftmargin=*]
                                   \item The user can switch between the two described modes
                                   \item The user can control the weather manually in "play mode"
                                   \item The user can choose the date and time in "real mode"
                               \end{itemize} \\ \hline
    \textbf{Fulfilled?}      & Yes \\ \hline
\end{tabularx}

\subsubsection{Performance}
\noindent\begin{tabularx}{\textwidth}{|l|X|}
    \hline
    \textbf{Case}            & T.6 \\ \hline
    \textbf{Test case}       & Performance \\ \hline
    \textbf{Expected result} & The shader code should run with reasonably good performance and should not show visual stutters or frame drops. \\ \hline
    \hline
    \textbf{Actual result}   & \vspace{-\topsep}\begin{itemize}[label={\checkmark},noitemsep,topsep=0pt,leftmargin=*]
                                   \item The simulation runs at approximately 60 \gls{fps} in Full HD
                                   \item There are no noticable frame drops
                                   \item There are no visual stutters
                               \end{itemize} \\ \hline
    \textbf{Fulfilled?}      & Yes \\ \hline
\end{tabularx}

\subsection{Visual Testing}

\subsubsection{Real Photographs}
\noindent\begin{tabularx}{\textwidth}{|l|X|}
    \hline
    \textbf{Case}            & T.7 \\ \hline
    \textbf{Test case}       & Real photographs \\ \hline
    \textbf{Expected result} & The visual output of the weather rendering system is to be compared with live weather photographs from \emph{Roundshot} cameras. The rendered image should resemble the weather of that time, to a reasonable extent. \\ \hline
    \hline
    \textbf{Actual result}   & \vspace{-\topsep}\begin{itemize}[label={\checkmark},noitemsep,topsep=0pt,leftmargin=*]
                                   \item The visual output resembles the weather for the same time to a reasonable extent
                               \end{itemize}
                               \begin{itemize}[label={$\times$},noitemsep,topsep=0pt,leftmargin=*]
                                    \item Clouds of the family "cirrus" are missing
                                \end{itemize} \\ \hline
    \textbf{Fulfilled?}      & Partially \\ \hline
\end{tabularx}

\subsubsection{Similar Products}
\noindent\begin{tabularx}{\textwidth}{|l|X|}
    \hline
    \textbf{Case}            & T.8 \\ \hline
    \textbf{Test case}       & Similar products \\ \hline
    \textbf{Expected result} & The visual output of the weather rendering system is to be compared with the in-game footage of Microsoft's \emph{Flight Simulator} game. The rendering system should achieve similar results, to a reasonable extent. \\ \hline
    \hline
    \textbf{Actual result}   & \vspace{-\topsep}\begin{itemize}[label={\checkmark},noitemsep,topsep=0pt,leftmargin=*]
                                   \item The visual output resembles that of Microsoft's Flight Simulator for similar weather conditions
                               \end{itemize} \\ \hline
    \textbf{Fulfilled?}      & Partially \\ \hline
\end{tabularx}
