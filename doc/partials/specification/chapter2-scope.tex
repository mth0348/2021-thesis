\section{Scope of Work}

\subsection{Initial Situation}
In computer graphics, especially in games, an astonishingly large group of features are reccurring across all programs and genres.
With the most obvious ones being water surfaces, cloudscapes and fire effects, they are present in almost any game. 
Naturally, those features grew in complexity, customizability and computational demands over time.
\\
One of the core mechanics for achieving realistic results is called a \emph{\gls{volumetric} \gls{shader}}.
A prototype such a \gls{shader} has been created in a previous project and will be used as base.

\subsubsection{Previous Work}
In a previous project, the process of creating a \gls{volumetric} \gls{shader} has already been researched and implemented in a prototype. Thanks to its high flexibility, different cloudscapes could be rendered by the same shader.

\begin{figure}[ht]
    \centering
        \begin{minipage}{0.47\linewidth}
            \includegraphics[width=\linewidth]{project2/project2-final.PNG}
            \captionof{figure}{Result of the previous work's shader (1).}
        \end{minipage}
    \hfill
        \begin{minipage}{0.47\linewidth}
            \includegraphics[width=\linewidth]{project2/project2-final2.PNG}
            \captionof{figure}{Result of the previous work's shader (2).}
        \end{minipage}  
\end{figure}

\noindent
During that project, some other important topics have been researched. Among those were \gls{volumetric} rendering, Perlin and Voronoi \gls{noisegeneration} algorithms, and a technique called \emph{\gls{raymarching}}.
\\
The implementations of those algorithms and methods will most likely be reused in this thesis and will be adapted and improved accordingly.

 \subsection{Goals}
\label{section:goals}
As the title of the thesis suggests, this work will primarily focus on clouds and cloudscapes.
The primary goal of the project is to research and implement rendering techniques for a real-time \gls{procedural} weather rendering system.
\\
The goals will be split into two distinct groups: mandatory and optional. However, this section only defines high-level goals. A detailed specification of all requirements can be found in \sectionref{section:requirements}.

\subsubsection{Mandatory Goals}
The following tasks must be accomplished during the project:
\begin{itemize}
    \item Understanding of different layers of clouds
    \item Understanding of compute shaders
    \item Implement a weather rendering system
    \item Incorporate real-time weather data from \emph{meteoblue}
\end{itemize}

\subsubsection{Optional Goals}
For further optional work, these tasks can be looked into:
\begin{itemize}
    \item Incorporate topological landscape models from \emph{swisstopo}
    \item Automatic validation of realism of rendered cloudscapes
    \item Automatic comparison of rendered cloudscapes and photographs
    \item Automatic categorization of rendered cloudscapes
    \item Performance optimization
\end{itemize}

\clearpage

\subsection{Vision}
This section defines a high-level vision for future work involving the results and implementations of this thesis. 
As listed in the primary goals, the weather rendering system will be based on compute shaders.
Compared to the prototype from the previous project, this is expected to result in a much better performance.
That in turn, allows for a more complex and realistic model.
\\
With the incorporation of real-time weather data and the use of topological landscape data, any given weather scenario could be simulated and rendered.
The desired outcome ideally looks similar to the image depicted in \autoref{img:rendered1}.
A rendered version of such a cloud system can look elusively realistic compared to an actual photograph, like in \autoref{img:photo1}.
\begin{figure}[ht]
    \centering
        \begin{minipage}{0.47\linewidth}
            \includegraphics[width=\linewidth]{msfs-ref1.jpg}
            \captionof{figure}{A rendered image of volumetric clouds \protect\cite{img:rendered:clouds01}.}
            \label{img:rendered1}
        \end{minipage}
    \hfill
        \begin{minipage}{0.47\linewidth}
            \includegraphics[width=\linewidth]{roundshot2021-01-18-12-50-cropped.png}
            \captionof{figure}{A photographic reference of clouds \protect\cite{img:photo:clouds01}.}
            \label{img:photo1}        
        \end{minipage}  
\end{figure}
\\
The first thought about the practical use of a fully-fledged volumetric cloud system might be a video game, since clouds are often a significant part of outdoor scenery in games.
However, for this thesis it is intended that the knowledge and results acquired during the given period will be used to recreate a lifelike weather system instead.
\\
To accurately reflect a weather system, conditions like precipitation, wind and cloudiness will be considered. This data is obtained from the firm \emph{meteoblue}.

\subsection{Educational Objectives}
Educational objectives include \gls{shader} programming, knowledge about \gls{computeshader}s, rendering techniques, common algorithms used in computer graphics like \gls{noisegeneration}, a general understanding of aspects needed to create a complete weather system and finally the incorporation of real-time data from a third party.

\subsection{Used Software and Tools}
All documentation will be written in \gls{latex} with Visual Studio Code.
The \gls{shader} will be implemented in Unity. The chosen \gls{shader} language is \gls{hlsl}.
For the presentation, Microsoft PowerPoint will be used.