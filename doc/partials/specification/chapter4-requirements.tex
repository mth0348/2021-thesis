\section{Requirements}
\label{section:requirements}
All requirements are grouped by type. This results in two major groups, which are research and development requirements. Each one of the requirements is derived from a goal listed in \sectionref{section:goals}.

\subsection{Research requirements}
Each research requirement is denoted with the letter "R" followed by its number. They are sorted according to priority, from most important to least important.
\emptyline
\noindent\begin{tabularx}{\textwidth}{|l|X|}
    \hline
    \textbf{Number}     & R.\stepcounter{requirements}\arabic{requirements} \\ \hline
    \textbf{Name}       & Understanding the basic nature of clouds \\ \hline
    \textbf{Description}& In order to be able to recreate a realistically looking cloud shape, one has to examine and understand the way a cloud forms and disperses again first. \\ \hline
\end{tabularx}
\vspace{0.8cm}

\noindent\begin{tabularx}{\textwidth}{|l|X|}
    \hline
    \textbf{Number}     & R.\stepcounter{requirements}\arabic{requirements} \\ \hline
    \textbf{Name}       & Understanding of different layers of clouds \\ \hline
    \textbf{Description}& Among other characteristics, altitude, humidity and atmospheric pressure dictate the look and genus of a cloud. The goal is to decide which cloud types are required for a believable weather system. \\ \hline
\end{tabularx}
\vspace{0.8cm}

\noindent\begin{tabularx}{\linewidth}{|l|X|}
    \hline
    \textbf{Number}     & R.\stepcounter{requirements}\arabic{requirements} \\ \hline
    \textbf{Name}       & Understanding of compute shaders \\ \hline
    \textbf{Description}& Compute shaders proved to be a highly efficient tool when it comes to heavy calculations, like simulations. 
                          To improve performance and therefore allow for more a sophisticated weather system, compute shaders have to be researched. \\ \hline
\end{tabularx}
\vspace{0.8cm}

\pagebreak
\subsection{Development requirements}
\setcounter{requirements}{0}
\label{section:requirements:dev}
Each development requirement is denoted with the letter "D" followed by its number. They are sorted according to priority, from most important to least important.
\emptyline
\noindent\begin{tabularx}{\linewidth}{|l|X|}
    \hline
    \textbf{Number}     & D.\stepcounter{requirements}\arabic{requirements} \\ \hline
    \textbf{Name}       & Noise generation based on compute shaders \\ \hline
    \textbf{Description}& To make full use of the power of compute shaders, it is best to let them execute computationally demanding tasks. In this case, specifically \gls{noisegeneration}. \\ \hline
\end{tabularx}
\vspace{0.8cm}

\noindent\begin{tabularx}{\linewidth}{|l|X|}
    \hline
    \textbf{Number}     & D.\stepcounter{requirements}\arabic{requirements} \\ \hline
    \textbf{Name}       & Incorporation of real-time weather data from \emph{meteoblue} \\ \hline
    \textbf{Description}& In order to achieve a high degree of realism, real-time weather data will be used. \emph{Meteoblue} offers different data package contracts, of which the "basic\textunderscore1h" is to be acquired.
    \newline \newline The usage of the data package requires physical locations. the chosen locations are: 
    \begin{itemize}
        \item Bern, Switzerland
        \item Fribourg, Switzerland
        \item Solothurn, Switzerland
    \end{itemize} \\ \hline
\end{tabularx}
\vspace{0.8cm}

\noindent\begin{tabularx}{\linewidth}{|l|X|}
    \hline
    \textbf{Number}     & D.\stepcounter{requirements}\arabic{requirements} \\ \hline
    \textbf{Name}       & Incorporation of height model data from \emph{swisstopo} \\ \hline
    \textbf{Description}& The 3D height model data from \emph{swisstopo} will be downloaded and mapped into a compatible format for Unity. This is then used as a base for the scenery.
    \newline For texture layers, the satelite image data from \emph{swisstopo} will be used and mapped onto the 3D model. \\ \hline
\end{tabularx}
\vspace{0.8cm}

\noindent\begin{tabularx}{\linewidth}{|l|X|}
    \hline
    \textbf{Number}     & D.\stepcounter{requirements}\arabic{requirements} \\ \hline
    \textbf{Name}       & Periodcally store photographs of 360-degree cameras \\ \hline
    \textbf{Description}& A comparison of real-time weather data and an actual photographic reference from that date and time will prove to be useful. This is why the images from such cameras will be stored on a local file system.
    \newline \newline There are many camera systems that offer 360-degree footage free of charge. The chosen system is that of the company \emph{Seitz} called \emph{Roundshot} \cite{roundshot}, with these locations: 
    \begin{itemize}
        \item Roundshot camera Bantiger, Switzerland
        \item Roundshot camera Gurtenpark, Switzerland
    \end{itemize} \\ \hline
\end{tabularx}
\vspace{0.8cm}

\noindent\begin{tabularx}{\linewidth}{|l|X|}
    \hline
    \textbf{Number}     & D.\stepcounter{requirements}\arabic{requirements} \\ \hline
    \textbf{Name}       & Implement a weather rendering system \\ \hline
    \textbf{Description}& Finally, the weather system has to implemented, incorporating the external data sources and rendering images of cloudscapes.
    \newline The system should be controllable via practical parameters, like point in time. \\ \hline
\end{tabularx}
\vspace{0.8cm}