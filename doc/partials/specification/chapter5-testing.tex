\section{Testing}
The project will be implemented and tested in Unity.
For testing, the following test cases can be used to verify and evaluate the implementation.
They are split into two groups, separating the incorporation of external data with the implementation of the system.

\subsection{External data testing}

\noindent\begin{tabularx}{\textwidth}{|c|l|X|}
    \hline
    \textbf{Case} & \textbf{Test case} & \textbf{Expected result} \\ \hline
    T.\stepcounter{testcases}\arabic{testcases} & Weather data & The data from \emph{meteoblue} is incorporated into the weather rendering system. The data directly controls all related variables. \\ \hline
    T.\stepcounter{testcases}\arabic{testcases} & Terrain data & The data from \emph{swisstopo} is incorporated into the weather rendering system. The height model defines the terrain height map. The satellite images are used for texturing. \\ \hline
    T.\stepcounter{testcases}\arabic{testcases} & Photographic data & There is a feature that allows to overlay the \emph{Roundshot} photograph of the same time and date as the rendered image was created for. \\ \hline
\end{tabularx}

\subsection{Functional testing}

\begin{tabularx}{\textwidth}{|c|l|X|}
    \hline
    \textbf{Case} & \textbf{Test case} & \textbf{Expected result} \\ \hline
    T.\stepcounter{testcases}\arabic{testcases} & Code functionality & The code for the weather rendering system compiles and runs without error. \\ \hline
    T.\stepcounter{testcases}\arabic{testcases} & User interface & The user is able to switch between the two modes, "real-life" and "sandbox". The user is also able to control the weather system over the user interface accordingly. \\ \hline
    T.\stepcounter{testcases}\arabic{testcases} & Performance & The shader code should run with reasonably good performance and should not show visible stutters or frame drops. \\ \hline
\end{tabularx}