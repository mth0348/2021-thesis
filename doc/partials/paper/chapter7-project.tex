\section{Project Management}
\label{section:project}

\subsection{Schedule Comparison}
\label{section:project:schedule}
The following chart shows the original schedule (in \color{gray}grey\color{black}) with a side-by-side comparison with the actual time spent for each task (in \color{cyan}blue\color{black}). It indicates that the schedule was mostly met throughout the project.
\vspace{\baselineskip}

\begin{ganttchart}[
    vgrid={dotted},
    hgrid={draw=black!50, dotted},
    bar/.append style={fill=lightgray},
    x unit=0.65cm,
    ms/.append style={milestone/.append style={fill=gray, scale = 0.7, yshift=0.2cm}},
    ms2/.append style={milestone/.append style={fill=cyan, scale = 0.7, yshift=-0.2cm}},
    progress label text={},
    bar1/.append style={bar height=0.2},
    bar2/.append style={bar height=0.2},
    bar2/.append style={bar top shift=0.018cm},
    ]{1}{16}
    \gantttitle{Work weeks}{16} \\
    \gantttitlelist{1,...,16}{1} \\
    \ganttgroup{Organization}{1}{4} \\
    \ganttbar[bar1]{Specification}{1}{4}
    \ganttbar[bar2, bar/.append style={fill=cyan}]{}{1}{4} \\
    \ganttmilestone[ms]{Specification finished}{4}
    \ganttmilestone[ms2]{}{4} \\
    \ganttmilestone[ms]{Expert meeting 1}{4}
    \ganttmilestone[ms2]{}{4} \\
    \ganttgroup{Research}{5}{11} \\
    \ganttbar[bar1]{Documentation}{5}{11}
    \ganttbar[bar2, bar/.append style={fill=cyan}]{}{5}{14} \\
    \ganttmilestone[ms]{Interview with meteoblue}{6} \\
    \ganttgroup{Implementation}{7}{15} \\
    \ganttbar[bar1]{Impl: Data processing}{7}{8}
    \ganttbar[bar2, bar/.append style={fill=cyan}]{}{7}{8} \\
    \ganttbar[bar1]{Impl: Rendering system}{9}{13}
    \ganttbar[bar2, bar/.append style={fill=cyan}]{}{9}{14} \\
    \ganttbar[bar1]{Impl: User interface}{14}{15}
    \ganttbar[bar2, bar/.append style={fill=cyan}]{}{15}{15} \\
    \ganttmilestone[ms]{Progress evaluation}{11}
    \ganttmilestone[ms2]{}{11} \\
    \ganttmilestone[ms]{Expert meeting 2}{15}
    \ganttmilestone[ms2]{Expert meeting 2}{15} \\
    \ganttmilestone[ms]{Implementation finished}{15}
    \ganttmilestone[ms2]{}{15} \\
    \ganttgroup{Finalization}{16}{16} \\
    \ganttbar[bar1]{Finalizing and review}{16}{16}
    \ganttbar[bar2, bar/.append style={fill=cyan}]{}{16}{16}
\end{ganttchart}
\emptyline
The only major discrepancy between the planned and the actual schedule is the interview with \emph{meteoblue}, which was turned down by \emph{meteoblue}.
The interview would overstep the limited license that was arranged for this project.
This was unfortunate, but left more time for the documentation, which turned out to require more time and effort.

\subsection{Fulfillment of Requirements}
\label{section:project:requirements}
The following table shows the original requirements and wether or not they were met during the project.
\emptyline
\begin{tabularx}{\textwidth}{|l|X|c|}
    \hline
    \textbf{N°} & \textbf{Name}                                                         & \textbf{Is met?} \\ \hline
    R.1             & Understanding the basic nature of clouds                              & Yes                 \\ \hline
    R.2             & Understanding of different characteristics of clouds                  & Yes                 \\ \hline
    R.3             & Understanding of compute shaders                                      & Yes                 \\ \hline
    D.1             & Periodical acquirement of real-time weather data from \emph{meteoblue}& Yes                 \\ \hline
    D.2             & Periodical acquirement of photographs of 360-degree cameras           & Yes                 \\ \hline
    D.3             & Acquirement of elevation model data from \emph{swisstopo}             & Substituted                  \\ \hline
    D.4             & Noise generation based on compute shaders                             & Yes                 \\ \hline
    D.5.1           & Implementation of data aggregation and processing                     & Yes                 \\ \hline
    D.5.2           & Implementation of core rendering system                               & Yes                 \\ \hline
    D.5.3           & Implementation of user interface                                      & Yes                 \\ \hline
    O.1             & Rendering performance optimization                                    & Partially           \\ \hline
\end{tabularx}
\emptyline
Both requirements D.3 and O.1 are not or only partially met.

\subsubsection{Elevation Model Replacement}
Originally, it was planned to retrieve elevation model data from \emph{swisstopo}. This turned out to be more problematic than anticipated, having multiple file formats that are incompatible with Unity Engine.
This would have required a middleware to convert the files to a suitable format, which was also difficult to find.
Additionally, the elevation model data was split into tiles, which would have needed to be stitched together in-engine.
\emptyline
Luckily, during research, a Unity Engine plugin for \emph{ArcGIS} services developed by \emph{ESRI} was found.
This allowed for a quick setup without further mapping and tiling of geodata or arial images.
\\
Since the generated 3D models from the plugin were more than a suitable replacement for the \emph{swisstopo} data, the requirement D.3 was annuled, as there was now another, easier way to get 3D elevation data.
From then on, only the \emph{ArcGIS} plugin was used.

\subsubsection{Rendering Performance Optimization}
There have been several attempts at optimizing the rendering performance.
This includes the reduction of \gls{raymarching} and \gls{lightmarching} steps, the careful choice of scale and amount of octaves for 3D \gls{noisegeneration} as well as the avoidance of using heavy operations like \lstinline[language=HLSL]{pow(), exp(), log()}.
However, there was not sufficient time to delve deeper into performance optimization, but since the software is able to run with a decent \gls{framerate}, the issue was no longer pursued.

\subsection{Future Work}
\label{section:project:futurework}
\subsubsection{Rendering Capabilities}
Despite the considerable effort already put into lighting and illumination methods, there are still some features missing.
One of those are god rays, the volumetric light shafts that shine through gaps in clouds, giving the scene even more depth.
Other absent features are the sun's and moon's halo: A bright circle around the celestial body.

\subsubsection{Live Data Feed}
The resulting weather rendering system is still dependent on locally accumulated meteo data. 
A potential follow-up project could be an implementation of a live data feed system that continuously updated the weather rendering system.

\subsubsection{Simulation Game}
Alternatively, the project could be turned into a simulator game, similar to how Microsoft's Flight Simulator works.

\subsubsection{Meteorological Events}
There is also the option to include meteorological events like thunderstorms, blizzards, rainbows and many more natural phenomena.

\subsection{Project Conclusion}
\label{section:project:conclusion}
It is noteworthy that three more weeks were put into research.
This is mainly because new methods and algorithms have been continuously researched during prototyping, which resulted in constant documentation of those findings.
However, this did not conflict with the remainder of the schedule.
\\
Still, the total amount of time spent was about ten percent more than the originally estimated time budget of 320 hours.
This is probably due to the fact that there was quite some effort put into the final implementation.
\\
The project occurred during the same time as the global coronavirus lockdown restrictions, but was unhindered by that.
\emptyline
In summary, all project requirements were met, almost all milestones were completed in time and the final implementation turned out great, making this a successful and very informative project.