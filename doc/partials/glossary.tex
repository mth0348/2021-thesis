% Acronyms
\newglossaryentry{hlsl}{name={HLSL}, text={HLSL}, first={high-level shading language (HLSL)}, description={High-level shading language. Developed by microsoft, this is a standard shader language for DirectX used in graphics programming}}
\newglossaryentry{json}{name={JSON}, text={JSON}, first={Java-Script object notation (JSON)}, description={Java-Script object notation. A light-weight data format that is stored as human-readable text}}
\newglossaryentry{slpk}{name={SLPK}, text={SLPK}, first={ESRI scene layer package (SLPK)}, description={ESRI scene layer package A custom, web-optimized format used for files related to ESRI }}
\newglossaryentry{png}{name={PNG}, text={PNG}, first={portable network graphic (PNG)}, description={Portable network graphic. A common format for lossless compressed image files}}
\newglossaryentry{ui}{name={UI}, text={UI}, first={user interface (UI)}, description={User interface. The interface that allows the user to interact with the software}}
\newglossaryentry{gpu}{name={GPU}, text={GPU}, first={GPU}, description={Graphics processing unit. A piece of hardware designed to rapidly manipulate and alter memory, often intented for output to a display device}}
\newglossaryentry{wmo}{name={WMO}, text={WMO}, first={World Meteorological Organization (WMO)}, description={A specialized agency conducting atmospheric science, climatology, hydrology and geophysics}}
\newglossaryentry{rop}{name={ROP}, text={ROP}, first={render output pipeline (ROP)}, description={A component responsible for calculating the final pixel colors or depth values via specific matrix and vector operations}}

% Glossary entries
\newglossaryentry{latex}{name=LaTeX, description={ A high-quality document preparation system designed for the production of technical and scientific documentation }}
\newglossaryentry{noisegeneration}{name=Noise generation, text={noise generation}, description={ Noise generation is used to generate textures of one or more dimension with seemingly random smooth transitions from black to white (zero to one) }}
\newglossaryentry{volumetric}{name=Volumetric, text={volumetric}, description={ This describes a technique which takes a 3D volume of data and projects it to 2D. It is mostly used for transparent effects stored as a 3D image }}
\newglossaryentry{raymarching}{name=Ray marching, text={ray marching}, description={ Ray marching is a type of method to approximate the surface distance of a volumetric object, where a ray is cast into the volume and stepped forward until the surface is reached }}
\newglossaryentry{lightmarching}{name=Light marching, text={light marching}, description={ The same concept as \gls{raymarching}, but instead of being cast into the volume, it is cast towards the primary light source with a constant step }}
\newglossaryentry{billboard}{name=Billboard, text={billboard}, description={ A 2D image always facing towards the main camera }}
\newglossaryentry{worldspace}{name=World space, text={world space}, description={ Coordinates defined with respect to a global Cartesian coordinate system }}
\newglossaryentry{polymesh}{name=Polymesh, text={polymesh}, description={ A polymesh is a 3D model composed of polygons or triangles }}
\newglossaryentry{lowpoly}{name=Low poly, text={low poly}, description={ A 3D polymesh with a relatively low count of polygons }}
\newglossaryentry{scalarfield}{name=Scalar field, text={scalar field}, description={ A scalar field describes a typically three-dimensional grid of elements called \textit{voxels}, each containing a scalar value }}
\newglossaryentry{vectorfield}{name=Vector field, text={vector field}, description={ It is the same as a scalar field, except the voxels are vector values }}
\newglossaryentry{spheretracing}{name=Sphere tracing, text={sphere tracing}, description={ Sphere tracing describes an optimized algorithm of ray marching by using signed distance functions to approximate the surface distance of the volume }}
\newglossaryentry{sdf}{name=Signed distance function, text={signed distance function}, description={ A signed distance function, short SDF, returns a positive distance if the origin is outside the volume and a negative distance if it is inside the volume }}
\newglossaryentry{surfacenormal}{name=Surface normal, text={surface normal}, description={ A \textit{surface normal} or \textit{normal} is a vector which is perpendicular to a given geometry, like a triangle or polygon }}
\newglossaryentry{gradient}{name=Gradient, text={gradient}, description={ The \textit{gradient} denotes the direction of the greatest change of a scalar function }}
\newglossaryentry{penumbra}{name=Penumbra, text={penumbra}, description={ The partially shaded outer region of diffuse shadows. Also described as soft edges }}
\newglossaryentry{shapeblending}{name=Shape blending, text={shape blending}, description={ In SDFs, shapes can be seemingly blended together by returning a interpolated value of those distances }}
\newglossaryentry{ambientocclusion}{name=Ambient occlusion, text={ambient occlusion}, description={ Also known as contact shadows, this method darkens points in the scene that are not or only slightly exposed to the light and its environment }}
\newglossaryentry{noise}{name=Noise, text={noise}, description={ A randomly generated pattern, referring to \gls{procedural} pattern generation }}
\newglossaryentry{translucent}{name=Translucent, text={translucent}, description={ An object or substance that is translucent allows light to be passed through it, meaning it is rendered transparently to some degree }} 
\newglossaryentry{parameters}{name=Parameters, text={parameters}, description={ Shader variables exposed to the Unity Editor }} 
\newglossaryentry{sss}{name=Subsurface scattering, text={subsurface scattering}, description={ SSS is a mechanism of light transport in which light enters a translucent object, is scattered around and exits the material at a different point, resulting in illuminated areas where the material is thin }} 
\newglossaryentry{sunlightforwarding}{name=Sunlight forward scattering, text={sunlight forward scattering}, description={ The process of sunlight shining through and illuminating the clouds which cover the sun }} 
\newglossaryentry{sunlighttransmittance}{name=Sunlight transmittance, text={sunlight transmittance}, description={ In this matter, the same as \gls{sunlightforwarding} }} 
\newglossaryentry{cnn}{name=Convolutional neural network, text={convolutional neural network}, description={ A neural network that is able to classify images }} 
\newglossaryentry{gan}{name=Generative adverserial network, text={generative adverserial network}, description={ A set of two neural networks, where one generates images and the other tries to tell wether those images are real or generated }} 
\newglossaryentry{aabb}{name=Axis-aligned bounding box, text={axis-aligned bounding box}, description={ A non-rotated bounding box enclosing an object completely }} 
\newglossaryentry{shader}{name=Shader, text={shader}, description={ A piece of software which runs on the \gls{gpu}, rendering geometrically defined objects to the screen  }} 
\newglossaryentry{computeshader}{name=Compute shader, text={compute shader}, description={ A shader which runs on the GPU but outside of the default render pipeline }} 
\newglossaryentry{fbm}{name=Fractal Brownian motion, text={fractal Brownian motion}, description={ Different iterations of continuously more detailed noise layered on top of each other }} 
\newglossaryentry{fractalnoise}{name=Fractal noise, text={fractal noise}, description={ In this matter, the same as \gls{fbm} }} 
\newglossaryentry{procedural}{name=Procedural, text={procedural}, description={ Created solely with algorithms and independant of any prerequisites }}
\newglossaryentry{histogram}{name=Histogram, text={histogram}, description={ A graphical representation of data like brightness or color distribution of a given photograph }}
\newglossaryentry{csg}{name=Constructive solid geometry, text={constructive solid geometry}, description={ Short CSG, stands for combining primitive geometric objects with Boolean operators }}
\newglossaryentry{neuralnetwork}{name=Neural network, text={neural network}, description={ A series of algorithms that can recognize and categorize certain patterns in a given set of data }}
\newglossaryentry{interpolation}{name=Interpolation, text={interpolation}, description={ In mathematics, interpolation describes a method of estimating unknown values that fall between known values }}
\newglossaryentry{wrs}{name=Weather rendering system, text={weather rendering system}, description={ The Unity application that is implemented during this project. It takes in live data from a weather service and uses topological elevation models to create a weather simulation, which is then rendered and up for comparison with live photographs }}
\newglossaryentry{altitude}{name=Altitude, text={altitude}, description={ A vertical distance measurement, in this context specifically the distance from sea level to the given object }}
\newglossaryentry{watervapor}{name=Water vapor, text={water vapor}, description={ Evaporated water in a gaseous form }}
\newglossaryentry{desublimation}{name=Desublimation, text={desublimation}, description={ The process of gas transitioning to liquid without passing through the liquid phase }}
\newglossaryentry{halophenomenon}{name=Halo phenomenon, text={halo phenomenon}, description={ White or colored rings or arcs of light around the sun or the moon, produced by cirrostratus clouds }}
\newglossaryentry{cloudlet}{name=Cloudlet, text={cloudlet}, description={ Small, white, puffy clouds that come in large quantities, together forming a cloud of the cumulus family }}
\newglossaryentry{precipitation}{name=Precipitation, text={precipitation}, description={ Rainfall. The result of atmospheric water vapor that has been condensed and now falls from clouds }}
\newglossaryentry{convection}{name=Convection, text={convection}, description={ The process of warm air rising from the surface and cooling at higher altitude, of which the moisture is then condensed into clouds }}
\newglossaryentry{thermal}{name=Thermal, text={thermal}, description={ In relation with meteorology, the hot, rising air from convection is called "thermal" }}
\newglossaryentry{weatherfront}{name=Weather front, text={weather front}, description={ A boundary between to air masses, which differ in temperature, wind direction and humidity }}
\newglossaryentry{warmfront}{name=Warm front, text={warm front}, description={ A warm \gls{weatherfront}, the boundary of a mass of air that carries mild or warm air. When colliding with a \gls{coldfront}, \gls{precipitation} is often followed }}
\newglossaryentry{coldfront}{name=Cold front, text={cold front}, description={ A cold \gls{weatherfront}, the boundary of a mass of air that carries cold or cool air. When colliding with a \gls{warmfront}, \gls{precipitation} is often followed }}
\newglossaryentry{occludedfront}{name=Occluded front, text={occluded front}, description={ When a cold front overtakes a warm front, it pushes the warm air upwards (\gls{thermal}s). The moisture of the warm air condenses as it rises, creating \gls{watervapor}. This often results in clouds with \gls{precipitation} }}
\newglossaryentry{occlusion}{name=Occlusion, text={occlusion}, description={ In meteorology, the clash of a warm front and a cold front. See \gls{occludedfront} }}
\newglossaryentry{lerp}{name=Linear interpolation, text={linear interpolation}, description={ Simply put, linear interpolation describes a method of finding values inbetween two points on the same line }}
\newglossaryentry{particlesystem}{name=Particle system, text={particle system}, description={ In computer graphics, a particle system is a technique that continuously spawns and recycles objects. They are often used to reproduce fire or smoke effects, with small flame or dust textures as particles }}
\newglossaryentry{pseudorandom}{name=Pseudo-random, text={pseudo-random}, description={ A random number generated with a deterministic algorithm, meaning that the same input will always give the same output }}
\newglossaryentry{textureslice}{name=Texture slice, text={texture slice}, description={ A 2D texture extracted from a 3D texture for a given depth }}
\newglossaryentry{framebuffer}{name=Frame buffer, text={frame buffer}, description={ The buffer that stores pixels for each frame, from which the monitor constantly reads. The monitor then displays those pixels on the screen }}
\newglossaryentry{kernel}{name=Kernel, text={kernel}, description={ In \gls{computeshader}s, the kernel represents an entry point and defines the method that is executed for each thread group when running the \gls{computeshader} }}
\newglossaryentry{texel}{name=Texel, text={texel}, description={ Short for texture element, a single pixel of a 2D texture }}
\newglossaryentry{voxel}{name=Voxel, text={voxel}, description={ Short for volume element, a single element of a 3D array or 3D texture }}
\newglossaryentry{fragment}{name=Fragment, text={fragment}, description={ In computer graphics, a fragment is a single pixel on the screen that is processed by a \gls{fragmentshader} and given a color in the process, effectively rendering it }}
\newglossaryentry{fragmentshader}{name=Fragment shader, text={fragment shader}, description={ A \gls{shader} that processes single pixels, called \gls{fragment}s, calculates its color and outputs that to the \gls{framebuffer} }}
\newglossaryentry{rasterization}{name=Rasterization, text={rasterization}, description={ Rasterization describes the final step in rendering. It is the task of taking an image described in vector geometry and converting it into a raster image (a series of pixels) }}
\newglossaryentry{framerate}{name=Frame rate, text={frame rate}, description={ The rate at which a new image (called frame) appears on the display }}
\newglossaryentry{shadowpass}{name=Shadow pass, text={shadow pass}, description={ A second \gls{shader} pass that only calculates the shadow of its object }}
