\section{Testing}

\subsection{Testing of Prototypes}
The project will be implemented and tested in Unity. For testing, the following test cases can be used to verify and evaluate the implementation.
\vspace{\baselineskip}

\noindent\begin{tabularx}{\textwidth}{|c|l|X|}
    \hline
    \textbf{Case} & \textbf{Test case} & \textbf{Expected result} \\ \hline
    1 & Working code & The shader code contains a vertex function as well as a fragment function and compiles successfully.  \\ \hline
    2 & Shaded outcome & The rendered image corresponds with the outcome of the related prototype as described in \sectionref{section:requirements:dev}. \\ \hline
    3 & Visual comparison & The final rendered output should, to some extend, look similar to a photographic reference image, as seen in \autoref{img:photo1}. For comparison, photographs taken from installed Roundshot systems can be used. \\ \hline
    3 & Performance & The shader code should run with good performance and should not show visible stutters or frame drops.  \\ \hline
\end{tabularx}