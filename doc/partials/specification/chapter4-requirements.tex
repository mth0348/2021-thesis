\section{Requirements}
\label{section:requirements}
All requirements are grouped by type. This results in two major groups, which are research and development requirements. Each one of the requirements is derived from a goal listed in \sectionref{section:goals}.

\subsection{Research Requirements}
Each research requirement is denoted with the letter "R" followed by its number.
\emptyline
\noindent\begin{tabularx}{\textwidth}{|l|X|}
    \hline
    \textbf{Number}     & R.\stepcounter{requirements}\arabic{requirements} \\ \hline
    \textbf{Name}       & Understanding the basic nature of clouds \\ \hline
    \textbf{Description}& In order to be able to recreate a realistically looking cloud shape, one has to examine and understand the way a cloud forms and disperses again first. \\ \hline
\end{tabularx}
\vspace{0.8cm}

\noindent\begin{tabularx}{\textwidth}{|l|X|}
    \hline
    \textbf{Number}     & R.\stepcounter{requirements}\arabic{requirements} \\ \hline
    \textbf{Name}       & Understanding of different characteristics of clouds \\ \hline
    \textbf{Description}& Among other characteristics, altitude, humidity and atmospheric pressure dictate the look and genus of a cloud. The goal is to decide which cloud types are required for a believable weather system. \\ \hline
\end{tabularx}
\vspace{0.8cm}

\noindent\begin{tabularx}{\linewidth}{|l|X|}
    \hline
    \textbf{Number}     & R.\stepcounter{requirements}\arabic{requirements} \\ \hline
    \textbf{Name}       & Understanding of compute shaders \\ \hline
    \textbf{Description}& Compute shaders proved to be a highly efficient tool when it comes to heavy calculations, like simulations. 
                          To improve performance and therefore allow for a more sophisticated weather system, compute shaders have to be researched. \\ \hline
\end{tabularx}
\vspace{0.8cm}

\clearpage

\subsection{Development Requirements}
\setcounter{requirements}{0}
\label{section:requirements:dev}
Each development requirement is denoted with the letter "D" followed by its number.
\emptyline
\noindent\begin{tabularx}{\linewidth}{|l|X|}
    \hline
    \textbf{Number}     & D.\stepcounter{requirements}\arabic{requirements} \\ \hline
    \textbf{Name}       & Periodical acquirement of real-time weather data from \emph{meteoblue} \\ \hline
    \textbf{Description}& In order to achieve a high degree of realism, real-time weather data will be used. \emph{Meteoblue} offers different data package contracts, of which the "basic\textunderscore1h" is to be acquired. The data will be downloaded daily.
    \newline \newline The usage of the data package requires physical locations. The chosen locations are: 
    \begin{itemize}
        \item Bern, Switzerland
        \item Fribourg, Switzerland
        \item Solothurn, Switzerland
    \end{itemize} \\ \hline
\end{tabularx}
\vspace{0.8cm}

\noindent\begin{tabularx}{\linewidth}{|l|X|}
    \hline
    \textbf{Number}     & D.\stepcounter{requirements}\arabic{requirements} \\ \hline
    \textbf{Name}       & Periodical acquirement of photographs of 360-degree cameras \\ \hline
    \textbf{Description}& A comparison of real-time weather data with an actual photographic reference from that date and time will prove to be useful. Images from such cameras will be stored periodically on a local file system.
    \newline \newline The chosen system is that of the company \emph{Seitz} called \emph{Roundshot}, with these locations: 
    \begin{itemize}
        \item Roundshot camera Bantiger, Switzerland
        \item Roundshot camera Gurtenpark, Switzerland
    \end{itemize} \\ \hline
\end{tabularx}
\vspace{0.8cm}

\noindent\begin{tabularx}{\linewidth}{|l|X|}
    \hline
    \textbf{Number}     & D.\stepcounter{requirements}\arabic{requirements} \\ \hline
    \textbf{Name}       & Acquirement of elevation model data from \emph{swisstopo} \\ \hline
    \textbf{Description}& The 3D elevation model data from \emph{swisstopo} will be downloaded and mapped into a compatible format for Unity. This is then used as a base for the scenery.
    \newline For texture layers, aerial images from \emph{swisstopo} will be used and mapped onto the 3D model. \\ \hline
\end{tabularx}
\vspace{0.8cm}

\noindent\begin{tabularx}{\linewidth}{|l|X|}
    \hline
    \textbf{Number}     & D.\stepcounter{requirements}\arabic{requirements} \\ \hline
    \textbf{Name}       & Noise generation based on compute shaders \\ \hline
    \textbf{Description}& To make full use of the power of compute shaders, it is best to let them execute computationally demanding tasks. In this case, specifically \gls{noisegeneration}. \\ \hline
\end{tabularx}
\vspace{0.8cm}

\clearpage

\noindent
The implementation requirement D.\stepcounter{requirements}\arabic{requirements} is split into three sub-requirements, D.\arabic{requirements}.1, D.\arabic{requirements}.2 and D.\arabic{requirements}.3.
\\

\noindent\begin{tabularx}{\linewidth}{|l|X|}
    \hline
    \textbf{Number}     & D.\arabic{requirements}.1 \\ \hline
    \textbf{Name}       & Implementation of data aggregation and processing \\ \hline
    \textbf{Description}& The first part of the weather rendering system is the external data aggregation and processing. \\ \hline
\end{tabularx}
\vspace{0.8cm}

\noindent\begin{tabularx}{\linewidth}{|l|X|}
    \hline
    \textbf{Number}     & D.\arabic{requirements}.2 \\ \hline
    \textbf{Name}       & Implementation of core rendering system \\ \hline
    \textbf{Description}& The second part of the weather rendering system is the core of the system itself. This includes \gls{noisegeneration}, \gls{volumetric} rendering, terrain generation, and so on. \\ \hline
\end{tabularx}
\vspace{0.8cm}

\noindent\begin{tabularx}{\linewidth}{|l|X|}
    \hline
    \textbf{Number}     & D.\arabic{requirements}.3 \\ \hline
    \textbf{Name}       & Implementation of user interface \\ \hline
    \textbf{Description}& Finally, the user should be able to control the weather system with an intuitive \gls{ui}. The user can also switch between "live" and "play" mode, as described in \sectionref{section:vision:ui}.  \\ \hline
\end{tabularx}

\subsection{Optional Requirements}
\setcounter{requirements}{0}
\label{section:requirements:optional}
Gathered from \sectionref{section:goals:optional}, there is one optional goal.
\emptyline
\noindent\begin{tabularx}{\linewidth}{|l|X|}
    \hline
    \textbf{Number}     & O.\stepcounter{requirements}\arabic{requirements} \\ \hline
    \textbf{Name}       & Rendering performance optimization \\ \hline
    \textbf{Description}& This includes optimizing shader code, finding early exits for looping algorithms and reducing the overall workload of processing the external data during runtime. \\ \hline
\end{tabularx}

\subsection{Summary of Requirements}
\label{section:requirements:summary}
The requirements are each prioritized with a number from one (1) to three (3), with 1 being the highest priority and 3 being the lowest priority.
\emptyline
\noindent\begin{tabularx}{\textwidth}{|l|X|c|}
    \hline
    \textbf{Number} & \textbf{Name}                                                         & \textbf{Priority} \\ \hline
    R.1             & Understanding the basic nature of clouds                              & 1                 \\ \hline
    R.2             & Understanding of different characteristics of clouds                  & 1                 \\ \hline
    R.3             & Understanding of compute shaders                                      & 1                 \\ \hline
    D.1             & Periodical acquirement of real-time weather data from \emph{meteoblue}& 2                 \\ \hline
    D.2             & Periodical acquirement of photographs of 360-degree cameras           & 2                 \\ \hline
    D.3             & Acquirement of elevation model data from \emph{swisstopo}             & 2                 \\ \hline
    D.4             & Noise generation based on compute shaders                             & 1                 \\ \hline
    D.5.1           & Implementation of data aggregation and processing                     & 2                 \\ \hline
    D.5.2           & Implementation of core rendering system                               & 1                 \\ \hline
    D.5.3           & Implementation of user interface                                      & 3                 \\ \hline
    O.1             & Rendering performance optimization                                    & 3                 \\ \hline
\end{tabularx}